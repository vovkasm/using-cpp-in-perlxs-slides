\documentclass[utf8x]{beamer}
\usepackage[russian]{babel}
\usepackage{listings}

\usetheme{Frankfurt}

\title{Особенности создания XS-модулей на языке C++}
\author{Владимир Тимофеев}
\date{2013}

\begin{document}
\lstset{language=C}

\begin{frame}
    \titlepage
\end{frame}

\begin{frame}{Обзор}
    \tableofcontents
\end{frame}

\section{Введение}

\subsection{Почему XS}

\begin{frame}
    \frametitle{Почему XS?}
    Потому что мы хотим:
    \begin{itemize}[<+->]
        \item Увеличить производительность
        \item Уменьшить потребление памяти
        \item Использовать сторонние компоненты
    \end{itemize}
\end{frame}

\subsection{Почему C++}

\begin{frame}
    \frametitle{Почему C++?}
    XS -- кодогенератор для C. Но C++...
    \begin{itemize}[<+->]
        \item Похож на C
        \item Сравним по скорости и потреблению памяти
        \item + Инкапсуляция
        \item + STL
    \end{itemize}
\end{frame}

\section{Особенности}

\subsection{Как найти компилятор}

\begin{frame}
    \frametitle{Как найти компилятор}
\end{frame}

\subsection{Конфликты имен}

\begin{frame}[fragile]
    \frametitle{Конфликты имен}
    \begin{exampleblock}{Пример}
        \begin{lstlisting}
        #undef do_open
        #undef do_close
        \end{lstlisting}
    \end{exampleblock}
\end{frame}

\subsection{Трансляция C++ исключений в Perl-исключения}

\begin{frame}
    \frametitle{Трансляция C++ исключений в Perl-исключения}
\end{frame}

\end{document}
