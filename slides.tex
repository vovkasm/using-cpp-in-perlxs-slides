\documentclass[pdflatex,hyperref={unicode=true}]{beamer}
\usepackage{cmap}
\usepackage[english,russian]{babel}
\usepackage{listings}
\usepackage{bookmark}
\usepackage{tabu}
\usepackage{pifont}
\usepackage{setspace}

\usepackage{fontspec}
\usepackage{unicode-math}
\defaultfontfeatures{Ligatures=TeX}
\defaultfontfeatures[CMU Sans Serif]{Script=Cyrillic}
\setmainfont{CMU Sans Serif}
\setsansfont{CMU Sans Serif}
\setmonofont{CMU Typewriter Text}

\usetheme{Frankfurt}
\usefonttheme{professionalfonts}

\title{Особенности создания XS-модулей на языке \cpp}
\author{Владимир Тимофеев}
\date{2013}

\usepackage{color}
\definecolor{bluekeywords}{rgb}{0.13,0.13,1}
\definecolor{greencomments}{rgb}{0,0.5,0}
\definecolor{redstrings}{rgb}{0.9,0,0}
\definecolor{graylinenumbers}{rgb}{0.6,0.6,0.6}

\lstset{% listings parameters
    basicstyle=\small\ttfamily\setstretch{0.4},
    showspaces=false,
    showtabs=false,
    showstringspaces=false,
    breaklines=true,
    breakwhitespace=true,
    numbers=left,
    numbersep=5pt,
    numberstyle=\color{graylinenumbers}\tiny,
    stepnumber=2,
    commentstyle=\color{greencomments},
    keywordstyle=\color{bluekeywords},
    stringstyle=\color{redstrings},
    language=C }

\lstdefinestyle{PerlXS}{
    language=C++,
    morekeywords={MODULE,PACKAGE,DESTROY} }

\DeclareRobustCommand{\cpp}{
    \texorpdfstring{\hbox{C\hspace{-0.5ex}\protect\raisebox{0.5ex}{\protect\scalebox{0.67}{++}}}}{C++}
}

\begin{document}

\frame{\titlepage}

\frame{\tableofcontents[hideallsubsections]}

\section{Введение}

\frame{\tableofcontents[currentsection,hideothersubsections]}

\subsection{Почему XS}

\begin{frame}[t]
    \frametitle{Почему XS?}
    Потому что мы хотим:
    \begin{itemize}[<+->]
        \item Увеличить производительность
        \item Уменьшить потребление памяти
        \item Использовать сторонние компоненты
    \end{itemize}
\end{frame}

\subsection{Почему \cpp}

\begin{frame}[t]
    \frametitle{Почему \cpp?}
    XS -- кодогенератор для C. Но \cpp\ldots
    \begin{itemize}[<+->]
        \item Похож на C
        \item Сравним по скорости и потреблению памяти
        \item + Инкапсуляция
        \item + STL, Boost, \dots
    \end{itemize}
\end{frame}

\section{Пример}
\frame{\tableofcontents[currentsection,hideothersubsections]}

\subsection{Обзор файлов}
\begin{frame}[t,fragile]
    \frametitle{Пример}
    \begin{tabular}{ l | c | c | c | r }
        \only<1>{\ding{42}}Makefile.PL &
        \only<2>{\ding{42}}CppStat.pm &
        \only<3-5>{\ding{42}}CppStat.xs &
        \only<6>{\ding{42}}typemap &
        \only<7>{\ding{42}}perlobject.map \\
        \hline
    \end{tabular}
    \only<1>{ \lstinputlisting[language=Perl]{CppStat/Makefile.PL} }
    \only<2>{ \lstinputlisting[language=Perl]{CppStat/CppStat.pm} }
    \only<3>{ \lstinputlisting[language=C++, style=PerlXS, firstnumber=1, lastline=15]{CppStat/CppStat.xs} }
    \only<4>{ \lstinputlisting[language=C++, style=PerlXS, firstnumber=17, firstline=17, lastline=32]{CppStat/CppStat.xs} }
    \only<5>{ \lstinputlisting[language=C++, style=PerlXS, firstnumber=34, firstline=34]{CppStat/CppStat.xs} }
    \begin{onlyenv}<6>
        \begin{lstlisting}[language=C++,style=PerlXS]
TYPEMAP
CppStat *     O_OBJECT
        \end{lstlisting}
    \end{onlyenv}
    \begin{onlyenv}<7>
        \begin{minipage}[t][4cm][t]{\textwidth}
            \begin{itemize}
                \item Можно поискать в интернете
                \item или сгенерировать с помощью модуля \href{http://search.cpan.org/perldoc?ExtUtils::Typemap::ObjectMap}{ExtUtils::Typemap::ObjectMap} на CPAN
                \item или почитать perldoc perlxs
            \end{itemize}
        \end{minipage}
    \end{onlyenv} 
\end{frame}


\section{Как найти компилятор}
\frame{\tableofcontents[currentsection,hideothersubsections]}

\subsection{CC = g++}
\begin{frame}[t,fragile]
    \frametitle{CC = g++}
    \lstinputlisting[language=Perl]{CppStat/Makefile.PL}
    \begin{itemize}
        \item Только GCC
        \item Все файлы как \cpp 
    \end{itemize}
\end{frame}

\subsection{-x c++}
\begin{frame}[t,fragile]
    \frametitle{-x c++}
    \lstinputlisting[language=Perl]{CppStat2/Makefile.PL}
    \begin{itemize}
        \item Компилятор тот же, что для perl
        \item Компилятор должен понимать опцию -x (GCC,Clang)
        \item Все файлы как \cpp 
    \end{itemize}
\end{frame}

\subsection{ExtUtils::CppGuess}
\begin{frame}[t,fragile]
    \frametitle{ExtUtils::CppGuess}
    \lstinputlisting[language=Perl]{CppStat3/Makefile.PL}
    \begin{itemize}
        \item Компилятор тот же, что для perl
        \item GCC,Clang,MSVC
        \item Все файлы как \cpp 
    \end{itemize}
\end{frame}

\subsection{Module::Build::WithXSpp}
\begin{frame}[t]
    \frametitle{Module::Build::WithXSpp}
    Использует ExtUtils::CppGuess, но...
    \begin{itemize}
        \item XS++
        \item Module::Build
    \end{itemize}
\end{frame}

\subsection{Статистика}
\begin{frame}[t]
    \frametitle{Статистика}
    \begin{itemize}
        \item Всего на CPAN: $\sim120$
        \item CC = g++: $\sim50$
        \item ExtUtils::CppGuess: $13$
        \item Module::Build: $\sim15$
        \item Module::Build::WithXSpp: $6$
    \end{itemize}
\end{frame}

\section{Конфликты имен}
\frame{\tableofcontents[currentsection,hideothersubsections]}

\subsection{iostream}
\begin{frame}[t,fragile]
    \frametitle{Конфликты имен}
    \begin{lstlisting}[language=C++,style=PerlXS]
#ifdef __cplusplus
extern "C" {
#endif

#include "EXTERN.h"
#include "perl.h"

#ifdef __cplusplus
}
#endif

#include "XSUB.h"

#undef do_open
#undef do_close

#include <iostream>
    \end{lstlisting}
\end{frame}

\section{Исключения}
\frame{\tableofcontents[currentsection,hideothersubsections]}

\subsection{Обычная генерация Perl-исключений}
\begin{frame}[t,fragile]
    \frametitle{Просто и неправильно}
    MyClass.cc
    \begin{lstlisting}[language=C++,style=PerlXS]
void MyClass::method(int arg) {
    std::vector<int> tmp;
    if (arg < 0) croak("MyClass::method: arg should be positive");
    // ...
}
    \end{lstlisting}
    \begin{itemize}
        \item<2-| alert@2-> Утечка памяти
        \item<3-> Деструктор для tmp вызван не будет
        \item<4-> Потому что croak реализован через longjmp
    \end{itemize}
\end{frame}

\subsection{Наивная реализация \cpp-исключений}
\begin{frame}[t,fragile]
    \frametitle{Переходим к \cpp исключениям}
    MyClass.cc
    \begin{lstlisting}[language=C++,style=PerlXS]
#include <stdexcept>
void MyClass::method(int arg) {
    std::vector<int> tmp;
    if (arg < 0) throw std::runtime_error(
        "MyClass::method: arg should be positive"
        );
    // ...
}
    \end{lstlisting}
    \begin{itemize}
        \item Программа упадет
        \item Исключение надо еще перехватить
    \end{itemize}
\end{frame}

\begin{frame}[t,fragile]
    \frametitle{Переходим к \cpp исключениям}
    \begin{onlyenv}<1->
        MyModule.xs
        \begin{lstlisting}[language=C++,style=PerlXS]
void
MyClass::method(int arg);
        \end{lstlisting}
        Придется переписать и обернуть в try...catch.
    \end{onlyenv}
    \begin{onlyenv}<2->
        \begin{lstlisting}[language=C++,style=PerlXS]
void
MyClass::method(int arg)
CODE:
    try {
        THIS->method(arg);
    }
    catch (const std::exception& e) {
        croak("Ops: %s", e.what());
    }
    catch (...) {
        croak("Ops: unknown c++ exception");
    }
        \end{lstlisting}
    \end{onlyenv}
\end{frame}

\begin{frame}[t,fragile]
    \frametitle{Переходим к \cpp исключениям}
    \begin{lstlisting}[language=C++,style=PerlXS]
void
MyClass::method(int arg)
CODE:
    try {
        THIS->method(arg);
    }
    catch (const std::exception& e) {
        croak("Ops: %s", e.what());
    }
    catch (...) {
        croak("Ops: unknown c++ exception");
    }
    \end{lstlisting}
    \begin{itemize}
        \item<2-| alert@2-> И получим опять утечку памяти (хоть и мелкую)
        \item<3-> Именно в такой код наши вызовы оборачивает ExtUtils::XSpp
    \end{itemize}
\end{frame}

\subsection{\cpp-исключения без утечек}
\begin{frame}[t,fragile]
    \frametitle{Вот так не течет}
    \begin{lstlisting}[language=C++,style=PerlXS]
void
MyClass::method(int arg)
CODE:
    char errmsg[1024];
    errmsg[0] = '\0';
    try {
        THIS->method(arg);
    }
    catch (const std::exception& e) {
        snprintf(errmsg, 1024, "Ops: %s", e.what());
    }
    catch (...) {
        snprintf(errmsg, 1024, "Ops: unknown c++ exception");
    }
    if (errmsg[0]) {
        croak("%s", errmsg);
    }
    \end{lstlisting}
\end{frame}

\section{Заключение}
\frame{\tableofcontents[currentsection,hideothersubsections]}

\subsection{MakeMaker?}

\begin{frame}[t,fragile]
    \frametitle{Хотелось бы писать Makefile.PL так}
    \begin{lstlisting}[language=Perl]
use ExtUtils::MakeMaker;

WriteMakefile(
    NAME         => 'CppStat',
    VERSION_FROM => 'CppStat.pm',
    TYPEMAPS     => ['perlobject.map'],
    XS_LANG      => 'c++',
);
    \end{lstlisting}
\end{frame}

\subsection{exceptions}
\begin{frame}[t,fragile]
    \frametitle{Может быть XS может выглядеть так?}
    \begin{lstlisting}[language=C++,style=PerlXS]
void
MyClass::method(int arg)
CPPCATCH: std
    \end{lstlisting}
\end{frame}

\subsection{Выводы}
\begin{frame}[t]
    \frametitle{Итого}
    \begin{itemize}
        \item<1-> Писать на \cpp можно
        \item<2-> Есть поле для улучшений
        \item<3-> Вопросы?
        \item<4-> \url{mailto:vovkasm@gmail.com}
        \item<4-> Эта презентация доступна на\\
            \href{https://github.com/vovkasm/using-cpp-in-perlxs-slides}{github.com/vovkasm/using-cpp-in-perlxs-slides}
    \end{itemize}
\end{frame}

\end{document}
